\mysection{Settlements}{civilization-settlements}


  The moons of \mylink{Tartarus}{the-spheres} contain wide, unpopulated wastes filled with ruins, dangers, and uncovered riches; the people of the Five Dwarfs cluster in various settlements that dot the landscape.  Settlements are broken into four conceptual sizes:

\callout {

  \mybullet {
    \item \mybold{Tiny}  Thorps, dorfs, "the boondocks" (\LT 1,000 people).  A Tiny Settlement may have a few things people might trade for, and you'll probably sleep in a hayloft for a day or two at the most before someone chases you off.  Tiny settlements only deal in trade, and \myital{maybe} a few iron pieces.  They are generally superstitious and uneducated (see \mylink{Burn the Witch!}{mystic-complication-witch}).  \mybold{You cannot take Downtime in a Tiny settlement}.

    \item \mybold{Small (Iron)}
    Hamlets (1,000 - 10,000 people).   The "iron" settlements. Small settlements generally deal in iron - if you attempt to spend silver in a Small settlement, you might meet with some strange looks and attract unwanted attention.  Trying to spend gold in a Small settlement is pretty much impossible.


    \item \mybold{Medium (Silver)}
    Villages, wyches and small towns (10,000 - 100,000 people).  The "silver" settlements. Medium settlements usually deal in silver, though gold is sometimes used by the wealthy and iron used by the impoverished.  

    \item \mybold{Large (Gold)}
    Big towns and cities (100,000+ people).  The "gold" settlements.  Large settlements deal almost entirely in gold.  Trying to spend iron pieces in a metropolis would be met with disdain, like trying to pay your cab fare with pennies (and unlike modern civilization, no one is under any obligation to honor your tender if they don't feel like it).
  }

}
  The size of the settlement is important when it comes to longer rests, spending treasure, and finding \mylink{Gear}{gear}, \mylink{Services}{gear-services}, and \mylink{Hirelings}{gear-hirelings}.
