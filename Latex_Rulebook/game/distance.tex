\end{multicols*}
\mysection{Distance}{game-distance}
\myimage{game/Distance2Col}
\begin{multicols*}{2}\raggedcolumns


\ed{Abstract distances are a mechanic invented by David Black (to the best of my knowledge).}

Like Time, Distance is broken into abstract lengths: \mybold{Close}, \mybold{Nearby}, \mybold{Far-Away}, and \mybold{Distant}.  The following rules deal with distance in \mybold{Combat}; see the section on \mylink{Travel}{movement-travel} under \mylink{Movement}{movement} for rules on longer journeys.

\mybullet {
    \item \mybold{Close} is 1-2 meters away - near enough to engage in hand-to-hand fighting.
    \item \mybold{Nearby} is 3 to 10 meters or so - near enough where you could reasonably hit someone with something you're throwing.
    \item \mybold{Far-Away} is more than 10 meters up to 50 meters or so - far away enough that you'd have to run for several seconds to cover the distance.
    \item \mybold{Distant} is anything further than 50 meters, up to the \MAX distance of 100 meters - far away enough that someone could conceivably escape if you were chasing them.
}

\cbreak


\crunch{Hex Combat}{distance-hex-combat}

If you'd prefer a "crunchier", less abstract Combat, you can use the following hex ranges:

\mybullet {

    \item \mybold{Close:}  The center hex and all neighboring hexes are considered Close range.
    \item \mybold{Nearby:} Any spot 2-3 hexes away from a center hex are considered Nearby range.
    \item \mybold{Far Away:} Any spot 4-9 hexes away from a center hex are considered Far-Away range.
    \item \mybold{Distant:} Any spot 10+ hexes away from a center hex is Distant range.
}

Thus, if you were to move somewhere Nearby (a single Action), you could move up to 3 hexes away.  Moving to or targeting a Far Away (2 Actions) hex is limited to 6 hexes (essentially half-way to Distant). You can move somewhere Close (1 hex away) as a "free" movement (so you could move 1 hex and attack, for instance), but you can't use this as part of a move from Close to Nearby (that is, you can't move 4 hexes). The hex map allows the Arbiter to enforce lines of sight and obstacles, and can allow you to play a more tactical wargame.



