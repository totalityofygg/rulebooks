%%%%%%%%%%%%%%%%%%%%%%%%%%%%%%%%%%%%%%%%%%%%
%%% INSCRIPTION
%%%%%%%%%%%%%%%%%%%%%%%%%%%%%%%%%%%%%%%%%%%%



\mysection{Inscription}{research-inscription}


\callout {
    \mybold{Research Costs}

    \mytable{Y Y} {
        \thead{\# Pips} & \thead{Coin per Pip} \\
    } {
        1-5 & 100\FE \\
        6-10 & 100\AG \\
        11+ & 100\AU
    }
}


Incantations, symbols, and engravings; arcane calligraphy, runed manuscripts, symbols etched on papyri and clay tablets. Through Inscription, the skilled Philosopher captures a small part of the Void and binds it inside a written word. Practicing Inscription requires the Settlement where you're taking \mylink{Downtime}{downtime} to have a \mylink{Library}{gear-services}. Inscription only requires \mybold{Days} of Downtime, and covers:

\mybullet{
  \item \mylink{Transcription}{research-inscription-transcription}: Writing \mylink{Secrets}{arcana-wizardry-secrets} from and to \mylink{Grimoires and Fetishes}{inscription-grimoires-fetishes}.
  \item \mylink{Tattooing}{research-inscription-tattoo}: Branding and marking the flesh of living things with magical symbols.
  \item \mylink{Sigils}{research-inscription-sigils}: Etching and inscribing magical runes.
}

  \myimage{research/Inscription_1}

\mysubsection{Transcription}{research-inscription-transcription}

Transcription is used to copy Secrets from a Fetish into a Grimoire, or from a Grimoire onto a Fetish. You cannot inscribe Secrets from your own \mylink{skull}{arcana-wizardry-skull} onto a Grimoire or Fetish (your brain is in the way).

Additionally, you may attempt to carefully translate the Grimoires of other Sorcerers - but be careful! A mistake can lead to serious unpleasantness (see \mylink{Grimoire Traps}{grimoire-traps} below).



\myhighlight{Translation}{inscription-translation}

A \mylink{Grimoire}{grimoires} must be carefully cataloged, keyed, and decrypted in order to be read by another. Each Research Pip you spend increases the die you may roll in order to successfully translate a Grimoire. If you roll a Failure, you must roll on the \mylink{Grimoire Traps}{grimoire-traps} table - otherwise, you may successfully copy the contents of the foreign Grimoire onto a Fetish or into your own Grimoire. \mybold{You must translate the foreign Grimoire every time you wish to read it}, but when you do you may inscribe as many Secrets as you're able.

    \mytable{Y Y} {
        \thead{\# Pips} & \thead{Die} \\
    } {
        1 & d3 \\
        2 & d6 \\
        3 & d10 \\
        4 & d16 \\
        5 & d24 \\
        6 & \myital{Automatic Success} \\
    }

\myhighlight{Inscribing}{inscription-inscribing}

Each Secret you wish to Inscribe costs 1 Research Pip.

You can write a Secret from a Fetish, \mylink{Sorcerer's Skull}{arcana-wizardry-skull}, or untrapped foreign Grimoire (see above) to your Grimoire.  You can also write a Secret from your Grimoire to a Fetish. See the section on \mylink{Fetishes}{fetishes} below for more info.

You cannot copy a Fetish to another Fetish, and you cannot copy Secrets from \myital{your} skull to a Grimoire or Fetish (your brain is in the way).


\mysubsection{Tattooing}{research-inscription-tattoo}

Only the most skilled scriveners dare to etch the Void into the flesh of the willing. \mybold{Each Tattoo you wish to create costs 6 Research Pips}.  The most common Tattoos are below - invent more with the Arbiter!


\myhighlight{Dagger (Limb)}{sorcerer-tattoo-dagger}

Touching the dagger immediately brings it into your hand.  Place the dagger back on the limb returns it to being a tattoo.  The dagger is a normal iron dagger when it is separated from your body.

\myhighlight{Torch (Limb)}{sorcerer-tattoo-torch}

As dagger above.  The torch never goes out.

\myimage{research/Tattoo}

\cbreak

\myhighlight{Compass (Limb)}{sorcerer-tattoo-compass}

Always points towards true north.  

\myhighlight{Quill and Scroll (Back)}{sorcerer-tattoo-quill-scroll}

The quill can be removed as the Dagger, above - and given to someone else.  Anything that the other person writes with the quill will appear on the scroll tattooed on your flesh ... painfully.  The writing disappears in Minutes.

\myhighlight{Eye (Palm)}{sorcerer-tattoo-eye}

The tattooed can see through the eye as if it were one of their normal eyes.

\myhighlight{Rope (Neck, Arms, Legs, Torso}{sorcerer-tattoo-rope}

As Dagger above, but uncoils from the body into 25m of rope. Needs to be completely rewound around the person to return to being a tattoo.


\myhighlight{Mermaid (chest or neck)}{sorcerer-tattoo-mermaid}

In addition to the tattoo, the Researcher inscribes a \mylink{Wizard Sigil}{inscription-sigil-wizard} on a different non-living object (traditionally a ship in an open bottle or a scrimshawed whale bone). When the tattooed creature draws breath, they inhale the air immediately around the object marked with the Sigil, instead of the air (or lack thereof) surrounding themselves. This means that as long as the object isn't immersed in water, surrounded by poison gas, locked in an airtight container, etc. the tattooed creature can breathe as normal (even if under water, surrounded by poison gas, locked in an airtight container, etc.) Erasing the Wizard Sigil permanently breaks the connection between the tattoo and the object.

\newpage

\mysubsection{Sigils}{research-inscription-sigils}

Hieroglyphs, runes, seals, and marks - Sigils are magical writing inscribed onto the surfaces of objects. Sigils are very obvious - they glow slightly and exude an indefinable arcane mystery - but what a Sigil actually \myital{does} can only be guessed at (or with a successfully \mylink{Skill: Lore}{skill-lore} try). 

Some Sigils remain even after their magic is exhausted, meaning a Sigil you encounter may be "lifeless". Only the Researcher who created a Sigil or someone skilled in the \mylink{Whispers of Sun Wukong}{vulgate-whisper-sun-wukong} can \mybold{Erase} a Sigil once it's been inscribed. For a Researcher to Erase a Sigil, they need only place their finger on it and read its name aloud (note: the finger does not need to be attached to the hand, and the finger doesn't have to have flesh on it).  Only the Researcher who etched the Sigil knows its "true name", but you may be able to divine it through \mylink{research in a library}{research}, creative use of \mylink{Descrying}{occultism-descry}, torture, etc.

\mybold{Unless otherwise specified, a Sigil can only be placed on something that is generally immobile and not alive} - a door, a wall, the floor, a statue, etc.  The author is immune to any Sigils inscribed by themselves, where appropriate. Finally, in cases where Sigils may "activate" more than once before they become lifeless, they require Minutes to recharge themselves.

\myimage{research/Sigil1}

\cbreak


\myhighlight{Novice Sigils}{sigils-novice}

\callout{
    \begin{center}
    Novice Sigils cost 2 Research Pips
    \end{center}
}


\myanchor{\mybold{Chaining Sigil}}{inscription-sigil-chaining}  

A Chaining Sigil can be inscribed and "married" to another Sigil.  The Chaining Sigil must be created at the same time as its partner Sigil.  If a Chaining Sigil is Erased, the Sigil it is partnered with will immediately manifest.


\myanchor{\mybold{Talking Sigil}}{inscription-sigil-talking} 

When anyone comes Close to an object with a Talking Sigil on it, it will yell out a word at ear-splitting volume.  It will repeat this word for 1 real-world minute before becoming lifeless.  Each additional word or minute you want the Talking Sigil to speak, you must spend an additional 1 Research Pip.

\myanchor{\mybold{Warding Sigil}}{inscription-sigil-warding}

This glyph can only be placed on something that is closed: a book, a lock, a chest but not a sword in a scabbard or someone's mouth (Arbiter's discretion).  If the warded item is opened, the Sigil immediately becomes lifeless.

When anyone comes Close to an item with a Warding Sigil on it, they must \SAVE{Hexes} or find themselves compelled to ignore the object - put the book down, leave the lock where it is, walk away from the door, etc. The Warding Sigil has no effect on beings who are immune to the \mylink{Secrets of the Mind}{arcana-wizardry-secrets-alignment}. For each additional Research Pip you spend, the Warding Sigil inflicts a -1 penalty to the Save (maximum -4).

\myanchor{\mybold{Wizard Sigil}}{inscription-sigil-wizard}

The most basic Sigil, necessary to prepare a weapon, object, or person as a receptacle for magic. Each Wizard Sigil is as unique as a fingerprint; if you're familiar with the work of a certain Philosopher, you'll notice their Wizard Sigil right away.  Otherwise, you can identify the owner of a Sigil with a successful \mylink{Skill: Lore}{skill-lore} try at the Arbiter's discretion.

For an additional Research Pip, a Wizard Sigil can be made temporarily "portable". The Sigil will appear inscribed upon the palm of your hand until the end of the Session. By touching your hand to an object, you can transfer the Sigil to it, where it will become permanent and remain until Erased. If you don't transfer the Wizard Sigil by the end of the Session, it disappears.


\myhighlight{Warlock Sigils}{sigils-warlock}

\callout{
    \begin{center}
    Warlock Sigils cost 7 Research Pips
    \end{center}
}


\myanchor{\mybold{Cursing Sigil}}{inscription-sigil-cursing}

Working with someone versed in the arts of \mylink{Cunning}{cunning}, you may place a \mylink{Malison}{occultism-malison} into a Cursing Sigil. After you inscribe the Sigil, the occultist can place a \mylink{Lesser or Greater Curse}{cunning-curses} inside of it. 

When anyone comes Close to an object with the Sigil on it, they must \SAVE{Hexes} or fall victim to the curse. If there is more than 1 target, they must all Save.  The Sigil becomes lifeless after it delivers its curse.  For each additional Research Pip you spend, the Cursing Sigil inflicts a -1 penalty to the Save (maximum -4).


\myanchor{\mybold{Elemental Sigil}}{inscription-sigil-elemental}

When inscribing this Sigil, choose an elemental type (lightning, fire, wind, etc.) Any creature who comes Close to an Elemental Sigil triggers its effect, blasting the entire Close area with its elemental payload. The Sigil deals twice the number of Research Pips spent in damage (Save for half), and becomes lifeless after it explodes.

\myanchor{\mybold{Petrifying Sigil}}{inscription-sigil-petrifying}

All creatures who comes Close to the Petrifying Sigil must Save or become permanently \mylink{Paralyzed}{effect-paralyzed}. The effect lasts until the Sigil is Erased. You are aware of everything going on around you, but you do not age when you are in this state (most Mortals who are "rescued" from a Petrifying Sigil centuries after their mishap are hopelessly insane). The Sigil will affect a number of creatures equal to the number of Research Pips spent before becoming lifeless.

\myanchor{\mybold{Portal Sigil}}{inscription-sigil-portal}

Portal Sigils must be placed on doors, gates, or entrances (Arbiter's discretion). The Portal Sigil is inscribed on the threshold or top of one doorway, and a \mylink{Wizard Sigil}{inscription-sigil-wizard} is placed on another. Once the Wizard Sigil is permanently inscribed, the two doors are connected as if they are the same door - anyone stepping through one door will walk through the door on the other side, regardless of distance.  The doorway can be walked through in either direction a number of times equal to the Research Pips spent before it becomes lifeless; Erasing either Sigil also closes the door and renders the Portal Sigil lifeless.


\myhighlight{Archmage Sigils}{sigils-archmage}

\callout{
    \begin{center}
    Archmage Sigils cost 12 Research Pips
    \end{center}
}

\myimage{research/Sigil2}

\newpage

\myanchor{\mybold{Death Sigil}}{inscription-sigil-death}

Any unfortunate Mortal or creature of 2 or fewer \HD (or level 2 or less) who comes Close to a Death Sigil is instantly slain (Save negates). Unseelie and Unhallowed are unaffected. For every additional 2 Research Pips you spend, you may affect creatures of a higher \LVL or \HD:

\mytable{Y Y} {
    \thead{\# Pips} & \thead{\LVL or \HD} \\
} {
    12 & 2 \\
    14 & 3 \\
    16 & 4 \\
    18 & 5 \\
    20 & 6 \\
}

Once it manifests, the Sigil immediately becomes lifeless.


\myanchor{\mybold{Demonic Sigil}}{inscription-sigil-demonic}

With the help of a Spriggan who commands the \mylink{Fealty of the Damned}{forgotten-fealty-damned}, you place a 5 \HD Obliterated inside of the Sigil. For every 2 Research Pips you spend, you may increase the \HD by 1 (to a maximum of 7). The Spriggan must have the available \mybold{Sovereignty} to summon the Obliterated from the Void to be bound inside of the Sigil, and chooses the 3 Traits that the demon will have (see the section on \mylink{the Obliterated}{forgotten-obliterated} for more info).

Anyone other than the Sorcerer or Spriggan who comes Close to the Remembrance Sigil will cause the Obliterated to come forth - very pissed off - and immediately attack.  It will continue attacking as long as anyone is within Combat range (Close, Nearby, Far-Away, or Distant) - but it won't pursue anyone.  The Obliterated will return to its Sigil if there is no one to fight (this heals all damage and removes all negative effects).  If the Obliterated is slain, the Sigil will become lifeless.

\cbreak

\myanchor{\mybold{Obliterating Sigil}}{inscription-sigil-obliterating}

Anyone Close to an Obliterating Sigil must Save or randomly suffer one of the effects below:

\callout{\footnotesize{
\mynumlist {
  \item Any Secrets in the vicinity - in their mind, on a Fetish, in a Grimoire, etc - are obliterated;
  \item Every insignificant item (0 Burden) item they're holding is obliterated;
  \item The victim's armor and weapons are obliterated (magic items get an additional Save);
  \item All coins carried by the person (except in Hammerspace) are obliterated;
  \item The victim's Grit is obliterated, dropping to 0;
  \item The victim's Flesh is obliterated, prompting an immediate \DEATH try.
}}}

The Obliterating Sigil will activate a number of time equal to the Research Pips spent before becoming lifeless.


\myanchor{\mybold{Revisitation Sigil}}{inscription-sigil-revisitation}

The Revisitation Sigil is inscribed on the floor of a room (usually the Researcher's lab), and a Wizard's Sigil is temporarily placed on the body (see above). At any time during the Session, if the temporary Wizard Sigil is Erased, the Researcher (alone) will be teleported to the location of the Revisitation Sigil. The Researcher may continue to create and bind a Wizard Sigil each Session if they so choose; they may do this a number of times equal to the Research Pips spent, at which point the Revisitation Sigil becomes lifeless. 



\newpage
