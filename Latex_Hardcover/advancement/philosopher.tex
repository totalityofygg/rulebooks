 %%%%%%%%%%%%%%%%%%%%%%%%%%%%%%%%%%%%%%%%%%%%%%%%%%%%%%%%%%%%%%%%%%%%%%
  %%%%  PHILOSOPHER %%%%%%%%%%%%%%%%%%%%%%%%%%%%%%%%%%%%%%%%%%%%%%%%%%%%%%%
  %%%%%%%%%%%%%%%%%%%%%%%%%%%%%%%%%%%%%%%%%%%%%%%%%%%%%%%%%%%%%%%%%%%%%%

\mysubsection{Philosopher Virtues}{advancement-philosopher-virtues}

\myemph{Unless otherwise specified, you can take each Virtue only once.}


  \mytable{Y Y Y} {
    \thead{Daredevil (Level 2+)} & \thead{Heroic (Level 4+)} & \thead{Legendary (Level 7+)} \\
  } {
    Charms  & Anarchist & Kallisti \\
    Chymist\Asterisk & Conduit of Blood & Kismet III \\
    Discordian & Doctor & Maestro \\
    Kismet I & Esteemed & Patron \\
    Medicine & Kismet II &  Personality III \\
    Personality I & Knowledge is Power & Saves III \\
    Professor & Personality II &  Staff Magic: Archmage \\
    Sacraments  & Saves II  & Tutor \\
    Saves I  & Staff Magic: Warlock & - \\
    Scribe\Asterisk & Well Read & - \\
    Sorcerer\Asterisk & - & - \\
    Staff Magic: Novice & - & - \\
}




\callout {
    \Asterisk This Virtue can be taken once per \LVL. See details in description
}

\begin{multicols*}{2}


\myhighlight{Anarchist}{adv-philosopher-anarchist}

You may \mypg{Ride the Tiger}{cruces-blood-ride-the-tiger} in lieu of rolling Blood Dice at any time. Add +2 Blood Dice to the total pool of dice you can channel in this way (i.e. if you have the Virtue: Discordian, you may now channel up to 4 Blood Dice). Once per Session, you may turn a \mybold{Calamity} you suffer into a \mybold{Mishap} instead (though the Arcana still fails).

\myhighlight{Charms}{adv-philosopher-charms}

You can perform the \mypg{Vulgate of Charms}{vulgate-charms} at will.

\myhighlight{Chymist}{adv-philosopher-chymist}

Identical to the Virtue of the same name under the \mypg{Philosopher Trope}{trope-philosopher}. If you already know this Virtue, gain 1 Research Pip. You may choose this Virtue once per \LVL.

\myimage{advancement/Philosopher4}

\newpage

\myhighlight{Conduit of Blood}{adv-philosopher-conduit-blood}

Once per Session, when rolling your Blood \POOL, you can make every die a natural 6.  Note this might trigger a Mishap, Calamity, or Ruin (see the section on \mypg{Wizardry}{arcana-wizardry}).  If you invoke a Calamity or Ruin when using this Virtue, the spell \mybold{does not fail}, but you still must roll on the appropriate table.

\myhighlight{Discordian}{adv-philosopher-discordian}

   You may \mypg{Ride the Tiger}{cruces-blood-ride-the-tiger} in lieu of rolling Blood Dice at any time. You may channel up to 2 Blood Dice (instead of 1) in this way. Once per Session, you may ignore the effects of a \mybold{Mishap}.


\myhighlight{Doctor}{adv-philosopher-doctor}

Advance your \INGENUITY \DCUP. 

\myhighlight{Esteemed}{adv-philosopher-esteemed}

Your wisdom and sound judgment have earned you high respect in "civilized" society. People will often seek you out for your counsel, and enemies consider you a diplomat before they consider you a foe. Even among strangers you are afforded the benefit of the doubt (for example, if you were to be caught up in a mutiny or a crime you might be spared unless you were directly involved). If you are taking Downtime in a Medium or Large Settlement, consider the duration of your stay to be one level "cheaper" i.e. treat Months as if they were Weeks, Weeks as Days, and Days as free (staying for Years still incurs the same cost).

\myhighlight{Kallisti}{adv-philosopher-kallisti}

You may \mypg{Ride the Tiger}{cruces-blood-ride-the-tiger} in lieu of rolling Blood Dice at any time. Add +2 Blood Dice to the total pool of dice you can channel in this way (i.e. if you have the Virtue: Discordian and Virtue: Anarchist, you may now channel up to 6 Blood Dice). Once per Session, you may turn a \mybold{Ruin} you suffer into a \mybold{Calamity} instead.

\cbreak

\myhighlight{Kismet I-III}{adv-philosopher-kismet}

Advance \mybold{all} aspects of your \mybold{Kismet} - \DEATH, \INJURY, and \INSANITY - to the next named level.

\myhighlight{Knowledge is Power}{adv-philosopher-knowledge-power}

You can cast spells off of Fetishes using your Knowledge Die.

\myhighlight{Maestro}{adv-philosopher-maestro}

Advance your \INGENUITY \DCUP. You may work with the Arbiter to create your own \mypg{Leechcraft}{arcana-leechcraft}.


\myhighlight{Medicine}{adv-philosopher-medicine}

You can perform the \mypg{Vulgate of Medicine}{vulgate-medicine} during the Shopping Step of Downtime. 

\myhighlight{Patron}{adv-philosopher-patron}

Like Ningauble and Sheelba, young and eager adventurers find their way to your doorstep. During the Training Step of Downtime, an Adventurer seeks you out who is willing to run "errands" for you. They won't work for free and are roughly 1st or 2nd level; depending on how complex the task is, they will return to you in Days, Weeks, or Months (Arbiter's discretion) with whatever you asked for. They won't kill for you, except indirectly (they're not assassins). The Arbiter has final say on whether or not they succeed, but this should be a conversation between the two of you before you send them on their quest. If you betray them (by not paying them, for example) the Arbiter is encouraged to avenge them.

\myhighlight{Personality I-III}{adv-philosopher-personality}

Advance two \mybold{different} aspects of your \mybold{Personality} \DCUP.

\myhighlight{Professor}{adv-philosopher-professor}

Advance your \INGENUITY \DCUP.


\myhighlight{Sacraments}{adv-philosopher-sacraments}

You gain a single Grace die (d4) which allows you to perform the \mypg{Sacraments}{vulgate-sacraments}.

\myhighlight{Saves I-III}{adv-philosopher-saves}

Advance \mybold{all} Saves to the next named level (Defenseless to Preserved; Preserved to Protected; etc).

   \myimage{advancement/ManHat}

\myhighlight{Scribe}{adv-philosopher-scribe}

Identical to the Virtue of the same name under the Philosopher Trope. If you already know this Virtue, gain 1 Research Pip. You may choose this Virtue once per \LVL.

\myhighlight{Sorcerer}{adv-philosopher-sorcerer}

Gain +1 Blood \POOL and choose \mybold{one} of the following:

\mybullet {
    \item One \mypg{Secret}{arcana-wizardry} of your choosing is now inscribed in your skull (see \mypg{Sorcerer's Skull}{arcana-wizardry-skull} for more info). This must be a Secret in your possession (either on a Grimoire or Fetish). 
    \item Automatically scribe three (3) \myital{random} Secrets into your Grimoire, provided you have room for them (reminder: a Grimoire can hold up to 10 Secrets). You do not need to know the Virtue of \mypg{Scribe}{trope-philosopher} to write these Arcana. If you roll a Secret already in your Grimoire, roll again (\mypg{Appendix A}{appendix-a} has a random spell table). 
    \item Scribe one Secret of your choosing into your Grimoire, provided you have room (reminder: a Grimoire can hold up to 10 Secrets). You do not need to know the Virtue of Scribe to write this Arcana. 
}

You may choose this Virtue once per \LVL.

\myhighlight{Staff Magic: Archmage}{adv-philosopher-staff-archmage}

You may create a \mypg{Staff of the Archmage}{wonder-staff-magic} containing Archmage Powers. Note that this does not grant you the ability to imbue a staff with Novice or Warlock powers.

\myhighlight{Staff Magic: Novice}{adv-philosopher-staff-novice}

Identical to the Virtue of the same name under the Philosopher Trope.

\myhighlight{Staff Magic: Warlock}{adv-philosopher-staff-warlock}

You may create a \mypg{Warlock staff}{wonder-staff-magic} containing Warlock Powers. Note that this does not grant you the ability to imbue a staff with Novice powers.

\myhighlight{Tutor}{adv-philosopher-tutor}

An apprentice joins you.  They are \mypg{Henchmen}{gear-henchmen} with Cowardly morale; they expect nominal pay, room, and board - otherwise, they will need to make periodic Morale checks at the discretion of the Arbiter.  They have 4 Flesh and wear no armor. The apprentice can (a) carry as many Grimoires or Fetishes as you wish without it taking any inventory space; (b) grant you +3 Research Pips; (c) grant you +2 to your Leechcraft rolls; and (d) take any physical damage for you.  You can only have 1 apprentice at any time.

\newpage

\myhighlight{Well Read}{adv-philosopher-well-read}

Once per Session, you can declare something to be true because you read it in a book. The base chance of the thing actually being true is 3 in 6. There has to be a plausible way you could know about it from reading books (new discoveries, minor details, and personal secrets are unlikely). You don't know whether or not it is true right away; the Arbiter will roll when it matters. You might only be partially correct, but you will never be catastrophically wrong. If you declare that bugbears fear albino goats, they will either fear albino goats or be indifferent to albino goats. They won't be driven into a murderous rage by them. If you have access to a \mypg{Library}{gear-services} the base chance increases to 5 in 6.

\cbreak

\myimage{advancement/Philosopher5}



\end{multicols*}

