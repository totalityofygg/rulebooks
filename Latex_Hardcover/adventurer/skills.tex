\end{multicols*}
\begin{center}
\mysection{Skills}{adventurer-skills}

\summary {

    \begin{center}
    Skills encompass certain "real-world" things your Adventurer can do: Bushcraft, Eyeball, Listen, Lore, Math, Salt, and Travel.

    \myskip

    You start Trained in one of these skills (d8,+1) but Untrained in the rest (d4,+0). 

\myskip

\myital{Optional: write down your starting \mypg{Languages}{skills-languages} during this step}.

    \end{center}
}



\end{center}


\begin{multicols*}{2}\raggedcolumns



  \mybullet {
    \item \myanchor{\mybold{Bushcraft}}{skill-bushcraft} tracking; hunting; camping; finding water; survival; making fires; "ranger shit".
    \item \myanchor{\mybold{Eyeball}}{skill-eyeball} estimate something's value or weight; size somebody up ("can we take this guy?"); judge distance, quality, etc.
    \item \myanchor{\mybold{Listen}}{skill-listen}  listen at doors; remember something the Arbiter told you that you've forgotten.
    \item \myanchor{\mybold{Lore}}{skill-lore}  general ancient knowledge; esoteric religions; what the weird writing says; history of this magic item; whether or not something is a component.
    \item \myanchor{\mybold{Math}}{skill-math} architecture; sloping rooms; construction; angles; calculations; strategy.
    \item \myanchor{\mybold{Salt}}{skill-salt} swimming; sailing boats; having sea-legs; fishing; navigating by the stars; tying knots.
    \item \myanchor{\mybold{Travel}}{skill-travel} estimating food and water for a convoy; riding horses and camels; mountain climbing.
  }


You roll a Skill try \myital{to see if something is possible and achievable}. The Arbiter should only call for a Skill try when the difference between success and failure would be interesting, or when a skill that's not interesting enough to describe shouldn't be an automatic success.  You can certainly suggest a Skill try to the Arbiter, but the ultimate decision rests with her.

\cbreak

\formula{Skill Try}{
    \RO: Skill Die \PLUS Bonus \PLUS \mybold{d20} \PLUS Modifiers 
}

Each Skill has a "name level" attached to it (Untrained, Trained, Skilled, Master, and Artist) that describes how rad you are at using the Skill. You can only increase your Skills by rolling \mybold{Failures}.

  \mytable{Y Y Y Y}{
    \thead{Rank} & \thead{Skill Die} & \thead{Bonus} & \thead{\small{Failures}} \\
  }{
      Untrained & d4 & +0 & 0 \\
      Trained  & d8 & +1 & 10 \\
      Skilled & d12 & +2 & 15 \\
      Master  & d20 & +3 & 25 \\
      Artist  & d24 & +4 & 50 \\
  }


\mysubsection{Possibility}{skills-possibility}

A \RO Skill try doesn't just determine success, but \myital{whether or not something is even possible}.  

If you don't \RO when interpreting those twisting eldritch runes (Skill: Lore), it means that it's simply not possible in these conditions.  Maybe it's your fault, maybe it isn't.  The runes can't be made out, or this is only a fragment, or whatever.  Who knows.  It's just not possible - you tried, failed, and now you know.

If you have a sailor adventuring with you who is Skilled (d12,+2) in Skill: Salt, they're going to have a better chance of success.  But if someone fails a Skill: Salt try before the sailor had a chance to try, then they fucked it all up (they didn't shiver the timbers or hoist the mainsail or forgot about high tide or what-have-you).  They should have waited for the professional to give it a shot.  This might be a permanent or temporary condition at the Arbiter's discretion.

After the first Adventurer succeeds, everyone else is able to follow along behind them.  Additional skill tries won't have any effect on whether something is possible or not.  If Adventurer 1 succeeds, it's possible - Adventurer 2 failing doesn't suddenly make it impossible.  If further Skill tries are necessary, everyone who follows after the first Adventurer adds that Adventurer's Bonus to their Skill try.


\mysubsection{Learning}{skills-learning}

Keep track of the number of \mybold{Failures} (a 1 or a 2) you roll with you Skill die (tally marks, numbers, etc). When you get enough failures to move you up to the next name level, increase your Skill and reset the failure counter for that Skill. It's important to note that \mybold{you can succeed} (roll a 20 or better) and still roll a \mybold{Failure} with your Skill die! Whether you succeed or fail at your \RO attempt, if you roll a Failure with the Skill die, increment the number of Failures.

\example{Oona is Untrained (d4,+0) in Skill: Listen, and has failed using this skill 9 times.  The Band is supposed to deliver a severed hand to someone, but no one can remember the name because no one wrote it down. Oona asks the Arbiter if she can use her Skill: Listen to try to remember the name, and the Arbiter agrees. Oona rolls a 2 on the d4 (a Failure) and a 12 on the d20 - for a total of 14 . Since this is her 10th Failure trying to use the Skill: Listen, she moves the skill up to Trained (d8,+1), and sets the failure counter for Skill: Listen back to 0. The bad news is that no one can remember the name of the recipient of the severed hand, but the good news is that Oona has a better chance of remembering something next time.}

  \mysubsection{Helping Out}{skills-helping-out}

  Maybe the runes are a little bit harder to decipher than you thought; maybe it's really important you not get lost in the woods right now.  If you want to hedge your bets a little, you can ask for help from \mybold{one} other Adventurer.  The helper adds their Bonus (+1 to +4) to your try (this is in addition to any other modifiers).  Again, only \mybold{one} Adventurer can help you out, unless they are a Pooka (in which case, they don't count against your one person).


  If your try is a \mybold{Failure} and the helper has a \myital{higher} skill level than you, you get credit for the failure (you're learning from someone more experienced, after all), but if you roll a \mybold{Failure} on your try and the helper has an \myital{equal or lower} skill level than your own, you get nothing. Only the Adventurer rolling the dice gets credit for a failure (the helper gets nothing).


  The person helping has to describe to the Arbiter how they're helping you; if it doesn't sound feasible, the bonus doesn't get applied.

  \example {
      Tony "Two Touch" Tomlinnson is Trained (d8,+1) at Bushcraft, and is traveling with the ranger Kurt Eberwulf, who is Skilled (d12,+2) at it. They are tracking a "friend", and the Arbiter tells Kurt and Tony that the tracks lead faintly into the deep wood. Tony asks the Arbiter if he can use Skill: Bushcraft to follow the tracks, and Kurt offers to help. The Arbiter agrees. 

\myskip

Tony needs to \RO using a d20 + d8+1 (his Bushcraft skill) + 2 (for Kurt's help). He fails with an 16 (an 11 on the d20, a 2 on the d8, and +3 in bonuses). Because Kurt has a higher skill than Tony does, he marks down a failure next to Bushcraft (since he rolled a 2). If Kurt and Tony had both been Trained (d8,+1), Tony wouldn't have been allowed to mark down the failure. Kurt grinds his teeth in frustration - it looks like Tony trampled on the tracks, and now no one can figure out where they might lead to.
}

  \mysubsection{Books}{skills-books}

   You may stumble across certain books that contain instructions, histories, theories, etc. that can assist you in making a Skill try.  If used as a reference, these books can provide a bonus between +1 (for a primer on Tea Ceremonies, say) to +4 (the definitive multi-volume history of Elfland) where appropriate. Books that can help you with Skills always have a \mypg{Burden}{gear-burden} of at least 1.

   Other books might allow you to learn a Skill during \mypg{Downtime}{downtime}.  In this case, the book can't bring your level higher than "Trained".


   \mysubsection{Environment}{skill-environment}

    The Arbiter may assign a modifier -4 to +4 modifier to the Skill try at their discretion based on the environment. Maybe the conditions are favorable (you're using Skill: Lore to read a tome while sitting in one of the great libraries of the capital cities) or unfavorable (you're using Skill: Lore to read a forgotten language by candlelight on a dungeon wall while being attacked). This is at the Arbiter's complete discretion.



  \mysubsection{Other Skills}{skills-other}

    The 7 Core Skills are the basic skills that everyone gets. If you write a killer background for your Adventurer, you can add another Skill to your character sheet at the Arbiter's discretion (you were a haberdasher in Lankhmar and now you have Skill: Tailoring or what have you). You may also find additional skills by reading books, apprenticeships, etc. (for example, you might find a book on teacups or something and you spend a Vacation reading it and now you have the Skill: Tea Ceremony).

    Here are a few other Skills mentioned in these rules.

  \mybullet {
    \item \myanchor{\mybold{Veins Lore:}}{skill-veins-lore} A Skill native to \mypg{the \Murk}{species-murk}. Any time you encounter something in the \mypg{Veins}{murk-veins-of-the-earth} you might not understand, you can try this Skill.  Some examples:  remembering what kind of gift is appropriate to give to an Ambassodile, what that smell is in the darkness, figuring out what kind of architect this Janeen is, etc.  If you are a Murk and were a slave of the culture you're rolling against, you get a +2 bonus.
    
    \item \myanchor{\mybold{Tinkering:}}{skill-tinkering} A Skill possessed by some \mypg{Night Children}{species-night-child} allowing you to fix, understand, and manipulate small devices and tools (think of it as a shittier version of certain \mypg{Whispers}{vulgate-whispers}). This Skill is most often used to disarm small mechanical traps or their triggers, pick simple locks, and to try to repair broken stuff.  If you fail your Skill: Tinkering try, the worst possible event happens i.e. the trap explodes for maximum damage, the weapon you're "fixing" is destroyed beyond repair, the improvised lockpick breaks and now no-one can pick the lock, etc.  You can use Skill: Tinkering during a Bivouac to try to repair a \UD of Armor, but if you fail you make it 1 \UD \mybold{worse}.

  }


\crunch{Languages}{skills-languages}

If your Arbiter is using languages in their game world, your Adventurer starts with a "common" \mypg{Dialect}{civilization-languages}. If the Adventurer is Mortal, they can speak one additional Dialect. If they are Unseelie, your secondary language(s) are:

\mylist {
  \item \mybold{Pooka:}  \mybold{Birdsong}.
  \item \mybold{Murk:}  \mybold{Silent Speech}.
  \item \mybold{Night Child:} \mybold{Draconic}.
  \item \mybold{Spriggan:} Spriggan can speak all dialects and idiolects.
}


Finally, \RO using your \INT + d12.  If you roll a 20, gain one additional language; 21, two languages; and 22, three additional languages. It's up to you if you can read AND write these languages.

