 \mysection{Money}{settlements-money}

  The Totality of Ygg subscribes to a Chartalist approach to economy; each moon's settlements are an amalgam of multiple civilizations built on the bones of the civilizations before them, stretching back 30,000 years or more.  The rarity of certain metals (silver and gold), as well as the utility of another (iron) has created an easy way to measure credit and debt.  A 5,000 year old gold coin found in a crypt and another minted by a current city have the same value; iron, silver, and gold are the uniform system for measuring credits and debts and have remained so for millennia.  

  Thus, coins are divided into 3 types - iron (\FE), silver (\AG), and gold (\AU) - and have the following value relative to one another

  \callout {
    \begin{center}
    100\FE \Equals 10\AG \Equals 1\AU (gold)
    \end{center}
  }

\myimage{civilization/Coins}

\cbreak

  \myimage{civilization/Moneychanger}

  Iron is the core of the monetary system, specifically the iron ingot (1kg of iron), which is equal to 100 iron coins.  It is not uncommon for iron coins to be smelted to forge the tools, weapons, and armor they pay for (particularly when various cities are trying to feed and arm their military).  

  \mybold{Ingots} of each type are common, and weigh 1kg each.  Every ingot you find is worth 100 of the coin type:

  \callout {
      \mybullet {
        \item 1 iron ingot \Equals 1kg iron \Equals 100\FE
        \item 1 silver ingot \Equals 1 kg silver \Equals 100\AG
        \item 1 gold ingot \Equals 1 kg gold \Equals 100\AU
      }

      4 1kg ingots / 400 coins count as 1 \mylink{Burden}{gear-burden}.
  }
